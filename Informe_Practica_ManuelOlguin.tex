%%%%%%%%%%%%%%%%%%%%%%%%%%%%%%%%%%%%%%%%%%%%%%%%%%%%%%%%%%%%%%
%%%%		PLANTILLA LATEX PARA INFORMES
%%%%			LATEX REPORT TEMPLATE
%%%%
%%%%	Autor	: Carlos Gonzalez Cortes
%%%%	Correo	: carlgonz@ug.uchile.cl
%%%%	Version	: 1.0
%%%%
%%%%	Notas	: Este codigo se entrega tal cual es y sin
%%%%			  ningun tipo de garantia. Sientase libre de
%%%%			  modificar y compartir.(acentos omitidos en
%%%%			  los comentarios por compatibilidad)
%%%%
%%%%%%%%%%%%%%%%%%%%%%%%%%%%%%%%%%%%%%%%%%%%%%%%%%%%%%%%%%%%%%




\documentclass[11pt,letterpaper]{article}
\usepackage[spanish]{babel}
%\usepackage[ansinew]{inputenc}
\usepackage[utf8]{inputenc}
% \usepackage[latin1]{inputenc}
\usepackage[letterpaper,includeheadfoot, top=0cm, bottom=1.5cm, right=2.0cm, left=2.0cm, headheight=77pt]{geometry}
\renewcommand{\familydefault}{\sfdefault}

\usepackage{graphicx}
\usepackage{color}
\usepackage{hyperref}
\usepackage{amssymb}
\usepackage{url}
%\usepackage{pdfpages}
\usepackage{fancyhdr}
\usepackage{hyperref}
\usepackage{subfig}

\usepackage{listings} %Codigo
\lstset{language=Python, tabsize=4,breaklines=true}

\definecolor{codegreen}{rgb}{0,0.6,0}
\definecolor{codegray}{rgb}{0.5,0.5,0.5}
\definecolor{codepurple}{rgb}{0.58,0,0.82}
\definecolor{backcolour}{rgb}{0.8, 0.8, 0.8}

\lstdefinestyle{mystyle}{
    %backgroundcolor=\color{backcolour},
    commentstyle=\color{codegreen},
    keywordstyle=\color{magenta},
    numberstyle=\tiny\color{codegray},
    stringstyle=\color{codepurple},
    basicstyle=\footnotesize,
    breakatwhitespace=false,
    breaklines=true,
    captionpos=t,
    keepspaces=true,
    numbers=left,
    numbersep=8pt,
    showspaces=false,
    showstringspaces=false,
    showtabs=false,
    tabsize=4
}

\usepackage{xcolor}

\colorlet{punct}{red!60!black}
\definecolor{background}{HTML}{EEEEEE}
\definecolor{delim}{RGB}{20,105,176}
\colorlet{numb}{magenta!60!black}

\lstdefinelanguage{json}{
    literate=
     *{0}{{{\color{numb}0}}}{1}
      {1}{{{\color{numb}1}}}{1}
      {2}{{{\color{numb}2}}}{1}
      {3}{{{\color{numb}3}}}{1}
      {4}{{{\color{numb}4}}}{1}
      {5}{{{\color{numb}5}}}{1}
      {6}{{{\color{numb}6}}}{1}
      {7}{{{\color{numb}7}}}{1}
      {8}{{{\color{numb}8}}}{1}
      {9}{{{\color{numb}9}}}{1}
      {:}{{{\color{punct}{:}}}}{1}
      {,}{{{\color{punct}{,}}}}{1}
      {\{}{{{\color{delim}{\{}}}}{1}
      {\}}{{{\color{delim}{\}}}}}{1}
      {[}{{{\color{delim}{[}}}}{1}
      {]}{{{\color{delim}{]}}}}{1},
}

\lstset{style=mystyle}


% footnote in footer
\newcommand{\fancyfootnotetext}[2]{%
  \fancypagestyle{dingens}{%
    \fancyfoot[LO,RE]{\parbox{12cm}{\footnotemark[#1]\footnotesize #2}}%
  }%
  \thispagestyle{dingens}%
}

\begin{document}
%\begin{sf}
% --------------- ---------PORTADA --------------------------------------------
\newpage
\pagestyle{fancy}
\fancyhf{}
%-------------------- CABECERA ---------------------
\fancyhead[L]{ \includegraphics[scale=0.3]{img/fcfm_dcc.pdf} }
%\fancyhead[R]{ \includegraphics[scale=0.4]{img/fcfm.png} }
%------------------ TÍTULO -----------------------
\vspace*{6cm}
\begin{center}
\Huge  {Informe de Práctica}\\
\vspace{1cm}
\small {CC4901 -- Práctica Profesional I}\\
\end{center}
%----------------- NOMBRES ------------------------
\vfill
\begin{flushright}
\begin{tabular}{ll}
Empresa: & NIC Chile Research Labs\\
Alumno: & Manuel Olguín\\
Carrera: & Ingeniería Civil en Computación\\
RUT:& 18.274.982 -- 6\\
E-Mail: & molguin@dcc.uchile.cl\\
Tel: & +56 9 7463 6997\\
& \today\\
& Santiago, Chile.
\end{tabular}
\end{flushright}

% ·············· ENCABEZADO - PIE DE PAGINA ············
\newpage
\pagestyle{fancy}
\fancyhf{}

%Encabezado
%\fancyhead[L]{\rightmark}
%\fancyhead[L]{\small \rm \textit{Sección \rightmark}} %Izquierda
%\fancyhead[R]{\small \rm \textbf{\thepage}} %Derecha

\fancyfoot[L]{\small \rm \textit{Sección \rightmark}} %Izquierda
\fancyfoot[R]{\small \rm \textbf{\thepage}} %Derecha

%\fancyfoot[L]{\small \rm \textit{Pie de página - Izquierda}} %Izquierda
%\fancyfoot[R]{\small \rm \textit{Pie de página - Derecha}} %Derecha
%\fancyfoot[C]{\thepage} %Centro

\renewcommand{\sectionmark}[1]{\markright{\thesection.\ #1}}
\renewcommand{\headrulewidth}{0.5pt}
\renewcommand{\footrulewidth}{0.5pt}

\newpage
\section{Certificado de la Empresa}
\newpage
\section{Observaciones}
\newpage
% =============== INDICE ===============

\tableofcontents
\listoffigures

% =============== SECCION ===============
\newpage
\section{Resumen}

El trabajo se realizó entre Agosto y Octubre del 2015 en NIC Chile Research Labs, el laboratorio de investigación y desarrollo en protocolos de internet de la Facultad de Ciencias Físicas y Matemáticas de la Universidad de Chile. \@ Consistió en
\begin{enumerate}
    \item rediseñar e actualizar la API del backend de BeCity, una aplicación Android desarrollada en el laboratorio, para que se adhiriera al modelo REST.
    \item diseñar e implementar un microservicio de manejo de imágenes para el proyecto SUR.
\end{enumerate}

BeCity es una aplicación móvil Android orientada a ciclistas, la cual pretende facilitar el tránsito del usuario por la ciudad. Cuenta con un servicio de búsqueda de rutas con ciclovías, y permite grabar los recorridos del usuario, guardando estadísticas de velocidad, distancia recorrida, calorías quemadas, etc. Funciona bajo el paradigna cliente-servidor: existe un servicio central (el backend) que almacena y maneja los datos, y la aplicación Android (el cliente) se comunica con este servicio para actualizar y obtener información. Esta comunicación se realiza mediante una API (Application Programming Interface) usando HTTP (HyperText Transfer Protocol), sobre la cual se realizó el trabajo previamente mencionado.\\

El rediseño de esta API tenía como fin su modernización y adaptación a estándares modernos, en específico el modelo REST (REpresentational State Transfer)\\

SUR (Southern URban Observatory) es una plataforma que pretende centralizar todos los proyectos desarrollados en NICLabs. En este sentido, SUR se plantea como un backend unificado para los distintos servicios, ofreciendo funcionalidades comunes como autentificación de usuarios y manejo de recursos como imágenes, y generando una integración más estrecha entre los servicios.\\

* explicar microservicio *
\newpage
\section{Introducción}
\subsection{Lugar de Trabajo}

NIC Chile Research Labs (en adelante, NICLabs)\cite{niclabs}, es el laboratorio de redes de la Facultad de Ciencias Físicas y Matemáticas de la Universidad de Chile. Fundado en 2007, sus principales áreas de investigación y desarrollo son protocolos de internet y seguridad informática.

El trabajo detallado en el presente informe se desarrolló entre octubre y diciembre del año 2015, de manera presencial en las instalaciones del laboratorio ubicadas en Av. Blanco Encalada 1975, Santiago, Chile.

\subsection{Equipo de Trabajo}

%Arreglar!%

NICLabs cuenta tanto con desarrolladores e investigadores de tiempo completo, así como con estudiantes de pre- y postgrado. En específico, el trabajo se desarrolló en una sala compartida con otras 6 personas de distintas especialidades (diseñadores, ingenieros y estudiantes de pregrado).

En cuanto a los proyectos descritos en el presente informe, ambos se encontraban bajo la dirección de Felipe Lalanne\cite{lalanne}, investigador y desarrollador experimentado. Cabe destacar que si bien existe un equipo de trabajo formalmente conformado para BeCity, éste sólo maneja y actualiza la versión actual del software, y por ende el practicante no se unió directamente a éste, ya que su trabajo consideraba el rediseño completo del backend.
Por otro lado, no existe grupo de trabajo para SUR, y el practicante trabajó directamente con el supervisor.

\subsection{Software y Conceptos Importantes}
En términos generales, ambos proyectos se desarrollaron en el lenguaje de programación Python\cite{python}, utilizando además JetBrains Pycharm 5 Professional Edition\cite{pycharm} como IDE (\emph{Integrated Development Environment} - Entorno de Desarrollo Integrado) principal.

\subsubsection{REST}

El modelo REST (por sus siglas en inglés, ``Representational State Transfer'') es una serie de restricciones arquitecturales que, al aplicarse a un servicio web, induce una serie de propiedades deseables (e.g. escalabilidad, estabilidad, rendimiento, etc). Al estar relacionado sólamente con la estructura arquitectural del software, no especifica detalles de implementación, y es independiente del lenguage de programación escogido.\\

A continuación, se expondrán brevemente las restricciones impuestas por el modelo REST:
\begin{enumerate}
    \item \textbf{Modelo Cliente - Servidor} \@ Debe existir una clara separación de responsabilidades mediante una interfaz uniforme. Por ejemplo, el servidor debe encargarse del almacenamiento y procesamiento de los datos, pero no de la representación visual de éstos, y vice-versa en el caso del cliente.
    \item \textbf{\textit{Statelessness} - Ausencia de Estados} \@ El servidor no debe guardar información del estado del cliente entre solicitudes. El estado es almacenado por el cliente; cada solicitud debe ser independiente y contener toda la información necesaria para realizarse.
    \item \textbf{Caché} \@ El servidor debe poder guardar solicitudes en caché para poder responder futuras solicitudes de manera más expedita.
    \item \textbf{Transparencia de Capas} \@ El cliente no debe poder determinar si está conectado directamente al servidor o a un servicio intermedio (proxy, etc).
    \item \textbf{Uniformidad de la Interfaz} \@ A su vez puede separarse en:
    \begin{itemize}
        \item \emph{Separación de recursos y su representación} \@ La representación externa de los recursos (datos) debe ser independiente de como son almacenados por el servidor. A su vez, estas representaciones contienen toda la información necesaria para poder modificar el recurso original.
        \item \emph{Identificación Uniforme de Recursos} \@ Por ejemplo, mediante URI's\cite{uri}.
        \item \emph{Mensajes Autodescriptivos} \@ Cada mensaje incluye información de cómo procesarse.
    \end{itemize}
\end{enumerate}

\subsubsection{API}

Una API (Application Programming Interface), en el contexto de servicios web, es una interfaz de interacción entre los componentes de un modelo servidor-cliente. Específicamente, consiste en un conjunto de puntos de enlace (endpoints) públicos en el servidor, a través de los cuales un cliente puede comunicarse e interactuar con el servicio central.

% explicar rest + http + api?

\subsubsection{PostgreSQL}

PostgreSQL\cite{postgres} es una de las principales implementaciones modernas del lenguaje SQL (Structured Query Language - Lenguaje Estructurado de Consultas) para bases de datos. Es un sistema de base de datos relacional ampliamente utilizado en la industria, y en específico, en BeCity se ocupa para el almacenamiento y consulta de los datos proporcionados por los usuarios.

\subsubsection{Python}

Python es un lenguaje de programación interpretado, dinámico y de alto nivel, con énfasis en la fácil lectura de su código y su brevedad sintáctica. Es un lenguaje multipropósito, apto para programas de pequeña y gran escala, y con soporte para múltiples paradigmas de programación (orientado a objetos, funcional, imperativo, etc).

En el contexto del trabajo realizado, Python, principalmente en su versión 2.7 (aunque en algunas instancias se ocupó también la versión más nueva del lenguaje, 3.5), fue el lenguaje escogido para el desarrollo de los servicios. Esto, ya que debido a su popularidad, cuenta con una gran cantidad de librerías bien documentadas y mantenidas que facilitaron el desarrollo. %Por ejemplo, entre las librerías utilizadas se encuentran:
%\begin{itemize}
%    \item \emph{Flask}\cite{flask}, la cual provee funcionalidades para el desarrollo de servicios web.
%    \item \emph{Restless}\cite{restless}, para implementar el modelo REST.
%    \item \emph{SQLAlchemy}\cite{sqlalchemy}, para interactuar con la base de datos.
%\end{itemize}

\subsubsection{JSON}

JSON (JavaScript Object Notation), es un estándar abierto de codificación de datos para la comunicación entre servidor y cliente, en la cual los datos se codifican en pares atributo-valor en ``cleartext'' (es decir, texto legible por humanos).
Por ejemplo, el siguiente extracto ilustra una posible codificación de la FCFM en JSON:
\begin{lstlisting}[language=JSON]
{
    "Nombre": "Facultad de Ciencias Fisicas y Matematicas",
    "Universidad": "Universidad de Chile",
    "Direccion": {
        "Calle": "Beauchef",
        "Numero": 850,
        "Comuna": "Santiago",
        "Region": "Metropolitana",
        "Pais": "CHILE"
    }
}
\end{lstlisting}

\subsubsection{Redis}



%\subsubsection{Docker}
\subsubsection{BeCity}

BeCity es una aplicación para smartphones con el sistema operativo Android, la cual tiene como fin la recolección de datos de ciclistas (por ejemplo, recorridos populares).

\subsubsection{SUR - Southern Urban Observatory}

\subsection{Situación Previa}
\subsubsection{BeCity}

En agosto del 2015, BeCity como aplicación ya se encontraba en funcionamiento, siendo distribuída a través de la tienda de aplicaciones de Google, la ``Play Store'', y con su backend montado en el servidor de NICLabs. Sin embargo, si bien contaba con una API completa para la comunicación entre clientes y backend, ésta no seguía el estándar REST, y se había decidido reconstruirla desde cero siguiendo éste estándar, para asegurar su futura escalabilidad y estabilidad.

\subsubsection{SUR}

Cuando el practicamente comenzó su trabajo en NICLabs, SUR como proyecto todavía no existía, por lo que todo el trabajo se desarrolló desde cero.

\subsection{Descripción General del Trabajo}

\subsubsection{BeCity}

El practicante diseñó e implementó parte de la nueva API, siguiendo las restricciones del modelo REST para servicios web. El trabajo se realizó por módulos de funcionalidad, siguiendo dos etapas por cada módulo:
\begin{enumerate}
    \item \emph{Diseño y Documentación} \\ En esta etapa, el practicante estudió la API existente del módulo escogido para entender el funcionamiento interno de éste, y los métodos que presentaba su interfaz. Luego, se diseñó y documentó la nueva versión del módulo, con especial énfasis en la nueva interfaz.
    \item \emph{Implementación y Testeo} \\ Luego de diseñado y documentado el módulo, se procedió a la implementación de las funcionalidades en código. Se utilizó el proceso de desarrollo ``Test Driven Development'', es decir, primero se escribían tests para funcionalidades deseadas, y luego se implementaban éstas.
\end{enumerate}

\subsubsection{SUR}

El trabajo realizado en SUR consistió en el desarrollo de un servicio de manejo de imágenes centralizado para todos los proyectos de la plataforma. El servicio debía básicamente almacenar y servir imágenes proporcionadas por los proyectos, por ejemplo para avatares de usuario. Se requería especial énfasis en la modularidad del servicio, de tal manera de que pudiese interactuar con cualquier otro proyecto que se adhiriera a la plataforma SUR en el futuro.

El trabajo consistió de dos etapas:
\begin{enumerate}
    \item \emph{Diseño del Servicio}\\ Para comenzar, se determinaron los requerimientos del servicio; las funcionalidades deseadas y la manera de interactuar con éste.
    \item \emph{Implementación}\\ Ya establecidos los requerimientos, se implementó el sistema deseado.
\end{enumerate}

\newpage
\section{Trabajo Realizado}

\subsection{BeCity}
\subsection{SUR}

Como se he mencionado anteriormente, el proyecto SUR pretende ser una plataforma central para todos los proyectos de INRIA Chile y NICLabs - esto con el fin de centralizar varios servicios que actualmente se encuentran ``duplicados'' en todos los proyectos (por ejemplo, manejo de usuarios y de imágenes). Un servicio central como SUR permitiría que cierta información se pudiese compartir de manera transparente entre los proyectos y servicios, permition por ejemplo que usuarios usen sus mismas credenciales de inicio de sesión, evitando la redundancia de tener que crear cuentas en cada uno de los servicios.
\\
Bajo esta lógica fue que al practicante se le solicitó la implementación del servicio central de imágenes, el cual permitiese compartir de manera transparente y eficiente recursos de imágenes (por ejemplo, avatares de usuario) entre los proyectos.

\subsubsection{Especificaciones}

El servicio en cuestión debía seguir una serie de especificaciones que aseguraban su eficiencia, escalabilidad y su compatibilidad con los demás servicios de la plataforma.

\begin{enumerate}
    \item \emph{Independencia:} \@ La implementación del servicio de imágenes debía ser independiente y agnóstica de la implementación del resto de los servicios. Es decir, si bien podía exister interdependencia entre el servicio de imágenes, y (por ejemplo) el servicio de autentificación, ésta no podía depender de la implementación de cada uno. Esto para asegurar que en el futuro, los servicios pudiesen actualizarse o cambiarse completamente sin perturbar el resto de los servicios en la plataforma.
    \item \emph{Adherencia al modelo REST}
\end{enumerate}

\newpage
\section{Conclusiones}
\newpage
\begin{thebibliography}{5}
    \bibitem{niclabs} \emph{NIC Chile Research Labs} - \url{http://niclabs.cl}

    \bibitem{lalanne} \emph{Felipe Lalanne} - \url{https://github.com/pipex}

    \bibitem{postgres} \emph{PostgreSQL - ``The world's most advanced open source database.''}\\ \url{http://www.postgresql.org/}

    \bibitem{python} \emph{The Python Programming Language} - \url{https://www.python.org/}

    \bibitem{pycharm} \emph{JetBrains Pycharm} - \url{https://www.jetbrains.com/pycharm/}

    \bibitem{uri} \emph{URI - Uniform Resource Identifier}: cadena de caracteres utilizada para identificar recursos específicos, siguiendo un esquema definido.\\ \url{https://en.wikipedia.org/wiki/Uniform_Resource_Identifier}

	\bibitem{flask} \emph{Flask is a microframework for Python based on Werkzeug, Jinja 2 and good intentions.}\\ \url{http://flask.pocoo.org/}

    \bibitem{restless} \emph{Restless - A lightweight REST miniframework for Python.}\\ \url{https://restless.readthedocs.org/en/latest/}

    \bibitem{sqlalchemy}\emph{SQLAlchemy - The Python SQL Toolkit and Object Relational Mapper}\\ \url{http://www.sqlalchemy.org/}
\end{thebibliography}


% ============= FIN DE DOCUMENTO ==============
\end{document}

% % ················ IMAGEN ·················
% \begin{figure}[ht!]
% \centering
% \fbox{\includegraphics[scale=0.6]{img/flujo.png}}
% \caption{Flujo de caja anual}\label{flujo}
% \end{figure}\\begin{enumerate}
    \item a
\end{enumerate}
% %··········································

% % ················ IMAGEN DOBLE ·················
% \begin{figure}[ht!] \centering
% \subfloat[Esquemático]{\includegraphics[scale=0.44]{img/seguidor.png}}
% \subfloat[Simulación]{\includegraphics[scale=0.45]{img/seguidor1.png}}
% \caption{Simulación como seguidor de voltaje}\label{seguidor}
% \end{figure}
% %··········································
